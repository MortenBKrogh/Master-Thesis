\documentclass[twoside, a4paper,12pt, openright]{memoir}

\usepackage[utf8]{inputenc}
\usepackage[english]{babel}

% Packages
\usepackage{geometry}
\usepackage{lipsum}% for auto generating text
\usepackage{afterpage}
\usepackage[export]{adjustbox}
\usepackage{etoolbox}
\usepackage{titlesec}
\usepackage{pdfpages}
\usepackage{lastpage}
\usepackage{fancyhdr}
%\usepackage[table,xcdraw]{xcolor}
\usepackage[flushleft]{threeparttable}
\usepackage{float}
\usepackage{amsmath}
\usepackage[labelfont=bf]{caption}

% Aarhus University Font
\usepackage{fontspec}
\defaultfontfeatures[AUFont]
    {
Extension = .ttf ,
UprightFont = ./Fonts/AUPassata_Rg,
BoldFont    = ./Fonts/AUPassata_Bold,
ItalicFont  = ./Fonts/AUPassata_Italic
}
\setmainfont{AUFont}
%\chapterfont{\fontfamily{Georgia}\selectfont}
%\setmainfont{Georgia} 


% Aarhus University Blue
\definecolor{AUdefault}{RGB}{0,61,133}

% ¤¤ Marginer ¤¤ %
\setlrmarginsandblock{3.5cm}{2.5cm}{*}		% \setlrmarginsandblock{Indbinding}{Kant}{Ratio}
\setulmarginsandblock{2.5cm}{3.0cm}{*}		% \setulmarginsandblock{Top}{Bund}{Ratio}
\checkandfixthelayout 						% Oversaetter vaerdier til brug for andre pakker

% Laver forskellige beregninger og sætter de almindelige længder op til brug ikke memoir pakker

%	¤¤ Afsnitsformatering ¤¤ %
\setlength{\parindent}{0mm}           	
% Størrelse af indryk
\setlength{\parskip}{3mm}          			
% Afstand mellem afsnit ved brug af double Enter
\linespread{1,5}											
% Linie afstand



% ¤¤ Indholdsfortegnelse ¤¤ %
\setsecnumdepth{subsection}		 			
% Dybden af nummerede overkrifter (part/chapter/section/subsection)
\maxsecnumdepth{subsection}					
% Ændring af dokumentklassens grænse for nummereringsdybde
\settocdepth{subsection} 								
% Dybden af indholdsfortegnelsen





%%%%%%% Opsætning af hyperlink til table of contents  %%%%
\usepackage{eso-pic}
% ¤¤ Visuelle referencer ¤¤ %
\usepackage[colorlinks]{hyperref}			% Danner klikbare referencer (hyperlinks) i dokumentet.
\hypersetup{colorlinks = true,				% Opsaetning af farvede hyperlinks (interne links, citeringer og URL)
    linkcolor = black,
    citecolor = black,
    urlcolor = black}

\usepackage{ifthen}
\newboolean{linktoc}
\setboolean{linktoc}{true}  %%% uncomment to show answers properly
%\setboolean{linktoc}{false}  %%% comment to show answers properly

\newcommand\AtPageUpperRight[1]{\AtPageUpperLeft{%
 \put(\LenToUnit{\paperwidth},\LenToUnit{-0.3\paperheight}){#1}%
 }}%
\newcommand\AtPageLowerRight[1]{\AtPageLowerLeft{%
 \put(\LenToUnit{\paperwidth},\LenToUnit{0.3\paperheight}){#1}%
 }}%

\ifthenelse{\boolean{linktoc}}%
{%
\AddToShipoutPictureBG{%
   \AtPageUpperRight{\put(-300,207){\hyperref[toc]{\textcolor{white}{Go to ToC}}}}
   %\AtPageLowerRight{\put(-70,-70){\hyperref[toc]{Go to ToC}}}
    }%
}%
{}%

\usepackage{titletoc}
\renewcommand{\contentsname}{\centering Contents}

\pretolerance=2500 							% Justering af afstand mellem ord (hoejt tal, mindre orddeling og mere luft mellem ord)


% ¤¤ Navngivning ¤¤ %
\addto\captionsdanish{
	\renewcommand\cftchaptername{\chaptername~}				% Skriver "Kapitel" foran kapitlerne i indholdsfortegnelsen
	\renewcommand\cftappendixname{\appendixname~}			% Skriver "Appendiks" foran appendiks i indholdsfortegnelsen
}

%%%% ORDDELING %%%%

\hyphenation{In-te-res-se e-le-ment}

\makeatletter
\renewcommand{\@biblabel}[1]{[#1]\hfill}
\makeatother









% ¤¤ Kapiteludssende ¤¤ %
\newif\ifchapternonum

\makeatletter
\let\ps@plain\ps@empty
\makeatother

\makechapterstyle{jenor}{					% Definerer kapiteludseende frem til ...

%\newfontfamily\subsubsectionfont[Color=AUdefault]{AUFont}

%\renewcommand\sectitlefont{\AUPLight\fontsize{15}{25}\selectfont\raggedleft}
  \renewcommand\beforechapskip{0pt}
  \renewcommand\printchaptername{}
  \renewcommand\printchapternum{}
  \renewcommand\printchapternonum{\chapternonumtrue}
  \renewcommand\chaptitlefont{\fontsize{25}{35}\selectfont\raggedleft}
  \renewcommand\chapnumfont{\fontsize{1.5in}{0in}\selectfont\color{AUdefault}}
   \renewcommand\printchaptertitle[1]{%
    \noindent
    \ifchapternonum
    \begin{tabularx}{\textwidth}{X}
    {\let\\\newline\chaptitlefont ##1\par} 
    \end{tabularx}
    \par\vskip-2.5mm\hrule
    \else
    \begin{tabularx}{\textwidth}{Xl}
    {\parbox[b]{\linewidth}{\chaptitlefont ##1}} & \raisebox{-15pt}{\chapnumfont \thechapter}
    \end{tabularx}
    \par\vskip2mm\hrule
    \fi
  }
}											% ... her

\chapterstyle{jenor}						% veelo Valg af kapiteludseende - Google 'memoir chapter styles' for alternativer

% ¤¤ Sidehoved/sidefod ¤¤ %
%\usepackage[page]{totalcount}

\makepagestyle{Uni}							% Definerer sidehoved og sidefod udseende frem til ...
\makepsmarks{Uni}{%
%	\createmark{chapter}{left}{shownumber}{}{. \ }
%	\createmark{section}{right}{shownumber}{}{. \ }
	\createplainmark{toc}{both}{\contentsname}
	\createplainmark{lof}{both}{\listfigurename}
	\createplainmark{lot}{both}{\listtablename}
%\createplainmark{bib}{both}{\bibname}
%	\createplainmark{index}{both}{\indexname}
%	\createplainmark{glossary}{both}{\glossaryname}
}
\nouppercaseheads											% Ingen Caps oenskes

% \makeevenhead{Uni}{Gruppe 2}{}{\leftmark}				
% Lige siders sidehoved (\makeevenhead{Navn}{Venstre}{Center}{Hoejre})
% \makeoddhead{Uni}{\rightmark}{}{Aalborg Universitet}			            % Ulige siders sidehoved (\makeoddhead{Navn}{Venstre}{Center}{Hoejre})
\makeevenfoot{Uni}{\thepage}{}{}							% Lige siders sidefod (\makeevenfoot{Navn}{Venstre}{Center}{Hoejre})
\makeoddfoot{Uni}{}{}{\thepage}								% Ulige siders sidefod (\makeoddfoot{Navn}{Venstre}{Center}{Hoejre})
%\makeheadrule{Uni}{\textwidth}{0.5pt}						% Tilfoejer en streg under sidehovedets indhold
%\makefootrule{Uni}{\textwidth}{0.5pt}{1mm}					% Tilfoejer en streg under sidefodens indhold

\copypagestyle{Unichap}{Uni}								% Sidehoved defineres som blank på kapitelsider
\makeoddhead{Unichap}{}{}{}
\makeevenhead{Unichap}{}{}{}
\makeheadrule{Unichap}{\textwidth}{0pt}
\aliaspagestyle{chapter}{Unichap}							% Den ny style vaelges til at gaelde for chapters
															% ... her
															
\pagestyle{Uni}												% Valg af sidehoved og sidefod (benyt "plain" for ingen sidehoved/fod)




\usepackage[natbibapa]{apacite}

%% You can pass in your own texcount params, e.g. -chinese to turn on Chinese mode, or -char to do a character count instead (which does NOT include spaces!)
%%% http://app.uio.no/ifi/texcount/documentation.html

%% To include references. DO NOT USE WITH BIBLATEX 
%TC:incbib

%% To include tabulars in main text count.
%TC:group table 0 1
%TC:group tabular 1 1

\newcommand{\detailtexcount}[1]{%
  \immediate\write18{texcount -merge -sum -q #1.tex output.bbl > #1.wcdetail }%
  \verbatiminput{#1.wcdetail}%
}

\newcommand{\quickwordcount}[1]{%
  \immediate\write18{texcount -1 -sum -merge -q #1.tex output.bbl > #1-words.sum }%
  \input{#1-words.sum} words%
}

%   -sum, -sum=   Make sum of all word and equation counts. May also use
%              -sum=#[,#] with up to 7 numbers to indicate how each of the
%              counts (text words, header words, caption words, #headers,
%              #floats, #inlined formulae, #displayed formulae) are summed.
%              The default sum (if only -sum is used) is the same as
%              -sum=1,1,1,0,0,1,1.


\newcommand{\quickcharcount}[1]{%
  \immediate\write18{texcount -1 -sum -merge -char -q #1.tex output.bbl > #1-chars.sum }%
  \input{#1-chars.sum} characters (not including spaces)%
}


\usepackage{neuralnetwork}
